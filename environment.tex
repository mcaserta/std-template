\section{Environment setup}

This section describes how to setup the system and the tools that make
up the environment where the \ac{AP} software is hosted. 

The components that we will be installing in this section are:

\begin{itemize}
  \item Ubuntu Linux Server 13.04
  \item \ac{JDK} 1.7.0 update 21
  \item \ac{SBT} 0.12.3
  \item Apache Maven 3.0.5
  \item Git 1.8.1.2
  \item MySQL 5.5
  \item Neo4j 1.9
  \item NodeJS 0.10.5
\end{itemize}

For each component we are indicating a minimum required version number,
although we suggest grabbing the latest available versions at the time
the installation is performed. If you are unsure about specific versions
compatibility, just email us at
\href{mailto:rocket-surgeons@acme.com}{\texttt{rocket-surgeons@acme.com}}.

Ubuntu and the \ac{JDK} are required to build and run the whole platform
while \ac{SBT}, Apache Maven and git are only required to build the
software. Additionally, Node.js is required to run the targeting user
interface software module.

In a hardened deployment setup, you might want to consider having a
dedicated compiler machine with all of the above environment tools set
up, then having different production machines with just the \ac{JDK}
installed to actually run the software.

\subsection{Ubuntu Linux Server}

A UNIX based \ac{OS} is required. We recommend Ubuntu Linux Server 13.04
for 64 bit machines since all of the following examples have been tested
on it, although they should work with little or no adaptation on any
Debian based Linux distribution. 

The command line examples should run on pretty much any UNIX variant,
including \ac{BSD} based systems such as Mac OS X. Distribution specific
package managers\footnote{such as \texttt{yum}, \texttt{rpm},
\texttt{portage}, \texttt{brew}, etc.} need of course to be used in such
case.

Ubuntu Linux Server might be downloaded from the official Ubuntu
download
site\footnote{\href{http://www.ubuntu.com/download/server}{\texttt{http://www.ubuntu.com/download/server}}}.
The file we used is called  \texttt{ubuntu-13.04-server-amd64.iso}.

For instructions on how to install and configure Ubuntu Linux Server,
please refer to the official online Ubuntu
documentation\footnote{\href{https://help.ubuntu.com/}{\texttt{https://help.ubuntu.com/}}}.

In our examples, we will refer to a regular user called \texttt{acme}
and assume all work to be done in her environment and home directory.
You can of course use a different user, as long as all work is done in
that user's environment. To add a new regular user and set her password,
please follow the example in code listing \ref{code:ubuntuuseradd}.

\begin{lstlisting}[label=code:ubuntuuseradd,caption=Adding a regular
user with the \texttt{acme} login]
# useradd --groups sudo --create-home acme
# passwd acme
\end{lstlisting}

We assume all binary package downloads to be placed in the
\texttt{\$HOME/dist} directory and the presence of a
\texttt{\$HOME/local} directory to install the downloaded packages. To
create these directories, please follow the example in code listing
\ref{code:ubuntuuserdirs}.

\begin{lstlisting}[label=code:ubuntuuserdirs,caption=Creating the \texttt{dist} and \texttt{local} directories]
$ mkdir ~/dist
$ mkdir ~/local
\end{lstlisting}

\subsection{Java Development Kit}

The \ac{JDK} must be downloaded from
\href{http://www.oracle.com/technetwork/java/javase/downloads/index.html}{Oracle's
web site}. The Linux x64 tarball\footnote{a \texttt{tar.gz} file.} is
the one we will be using\footnote{at the time of writing, the current
tarball is a file called \texttt{jdk-7u21-linux-x64.tar.gz}.}. To unpack
the tarball and verify its contents, please follow the example in code
listing \ref{code:jdkunpack}.

\begin{lstlisting}[label=code:jdkunpack,caption=Unpacking the tarball to the \texttt{local} directory and checking its contents]
$ tar xz -C ~/local -f ~/dist/jdk-7u21-linux-x64.tar.gz 
$ ls ~/local
jdk1.7.0_21
$ ls ~/local/jdk1.7.0_21
bin        lib          src.zip
COPYRIGHT  LICENSE      THIRDPARTYLICENSEREADME-JAVAFX.txt
db         man          THIRDPARTYLICENSEREADME.txt
include    README.html
jre        release
\end{lstlisting}
 
To make the java tools available to the command line prompt, the bottom
of your \texttt{\$HOME/.profile} file should look as in code listing
\ref{code:jdkpath}.

\begin{lstlisting}[label=code:jdkpath,caption=Code to add to the bottom of the \texttt{\$HOME/.profile} file]
JAVA_HOME="$HOME/local/jdk1.7.0_21"
PATH="$JAVA_HOME/bin:$PATH"
\end{lstlisting}

Having modified the \texttt{\$HOME/.profile} file, run the commands in
code listing \ref{code:jdksourcing} to make the changes available to the
current prompt and to verify that the java bytecode interpreter works.

\begin{lstlisting}[label=code:jdksourcing,caption=Sourcing the \texttt{\$HOME/.profile} file and verifying the java interpreter]
$ source ~/.profile
$ java -version
java version "1.7.0_21"
Java(TM) SE Runtime Environment (build 1.7.0_21-b11)
Java HotSpot(TM) 64-Bit Server VM (build 23.21-b01, mixed mode)  
\end{lstlisting}

\subsection{Simple Build Tool}
\subsubsection{Installation}

The \ac{SBT} tarball should be downloaded from the official download
page\footnote{\href{http://www.scala-sbt.org/download.html}{\texttt{http://www.scala-sbt.org/download.html}}}.
At the time of writing, the current version is 0.12.3.

To unpack the distribution tarball and check its contents, please run
the commands in code listing \ref{code:sbtunpack}.

\begin{lstlisting}[label=code:sbtunpack,caption=Unpacking the tarball to the \texttt{local} directory and checking its contents]
$ tar xz -C ~/local -f ~/dist/sbt.tgz 
$ ls ~/local
jdk1.7.0_21  sbt
$ ls ~/local/sbt
bin  jansi-license.txt
\end{lstlisting}

Modify the \texttt{\$HOME/.profile} file so that its bottom now looks
like the example in code listing \ref{code:sbtpath}.

\begin{lstlisting}[label=code:sbtpath,caption=Code to add to the bottom of the \texttt{.profile} file]
JAVA_HOME="$HOME/local/jdk1.7.0_21"
SBT_HOME="$HOME/local/sbt"
PATH="$SBT_HOME/bin:$JAVA_HOME/bin:$PATH"
\end{lstlisting}

Having modified the \texttt{\$HOME/.profile} file, run the commands in
code listing \ref{code:sbtverify} to make the changes available to the
current prompt and to verify that \ac{SBT} works:

\begin{lstlisting}[label=code:sbtverify,caption=Sourcing the \texttt{.profile} file and verifying SBT works]
$ source ~/.profile
$ sbt --version
sbt launcher version 0.12.3
\end{lstlisting}

\subsubsection{Configuration}

Make sure your shell environment defines an \texttt{SBT\_OPTS} variable
as shown in the example in code listing \ref{code:sbtoptsverify}.

\begin{lstlisting}[label=code:sbtoptsverify,caption=Verifying SBT options]
$ echo $SBT_OPTS                                                               
-XX:+CMSClassUnloadingEnabled -XX:MaxPermSize=256M
\end{lstlisting}

This can be achieved by adding the command as in code listing
\ref{code:sbtoptsset} to your \texttt{\$HOME/.profile}.

\begin{lstlisting}[label=code:sbtoptsset,caption=Setting SBT options in the \texttt{.profile} file]
SBT_OPTS='-XX:+CMSClassUnloadingEnabled -XX:MaxPermSize=256M'
\end{lstlisting}

Authentication needs to be configured in order to allow the \ac{SBT}
resolver to use the nexus repository manager. An
\texttt{\$HOME/.ivy2/.auth-acme} configuration file must be created
with contents such as in code listing \ref{code:sbtauth}. You should of
course overwrite the \texttt{user} and \texttt{password} fields with
sensible values.

\begin{lstlisting}[label=code:sbtauth,caption=Setting SBT authentication
options in the \texttt{\$HOME/.ivy2/.auth-acme} file]
realm=Sonatype Nexus Repository Manager
host=nexus.acme.com
user=username
password=secret
\end{lstlisting}

\subsection{Apache Maven}
\subsubsection{Installation}

The Apache Maven tarball should be downloaded from the official download
page\footnote{\href{http://maven.apache.org/download.cgi}{\texttt{http://maven.apache.org/download.cgi}}}.
At the time of writing, the current version is 3.0.5.

To unpack the distribution tarball and verify its contents, please run
the commands in code listing \ref{code:mvnunpack}.

\begin{lstlisting}[label=code:mvnunpack,caption=Unpacking the tarball to the \texttt{local} directory and checking its contents]
$ tar xz -C ~/local -f ~/dist/apache-maven-3.0.5-bin.tar.gz 
$ ls ~/local
apache-maven-3.0.5  jdk1.7.0_21  sbt
$ ls ~/local/apache-maven-3.0.5
bin  boot  conf  lib  LICENSE.txt  NOTICE.txt  README.txt
\end{lstlisting}

Modify the \texttt{\$HOME/.profile} file so that its bottom looks like
the example in code listing \ref{code:mvnpath}.

\begin{lstlisting}[label=code:mvnpath,caption=Code to add to the bottom of the \texttt{.profile} file]
JAVA_HOME="$HOME/local/jdk1.7.0_21"
SBT_HOME="$HOME/local/sbt"
MAVEN_HOME="$HOME/local/apache-maven-3.0.5"
PATH="$MAVEN_HOME/bin:$SBT_HOME/bin:$JAVA_HOME/bin:$PATH"
\end{lstlisting}

Having modified the \texttt{.profile} file, run the commands in code
listing \ref{code:mvnverify} to make the changes available to the
current prompt and to verify that maven works.

\begin{lstlisting}[label=code:mvnverify,caption=Sourcing the \texttt{.profile} file and verifying maven works]
$ source ~/.profile
$ mvn --version
Apache Maven 3.0.5 (r01de14724cdef164cd33c7c8c2fe155faf9602da;
Maven home: /home/acme/local/apache-maven-3.0.5
Java version: 1.7.0_21, vendor: Oracle Corporation
Java home: /home/acme/local/jdk1.7.0_21/jre
Default locale: en_US, platform encoding: UTF-8
OS name: "linux", version: "3.8.0-19-generic", arch: "amd64"
\end{lstlisting}

\subsubsection{Configuration}

A \texttt{\$HOME/.m2/settings.xml} file must be created with contents
such as in code listing \ref{code:mvnsettings}. The \texttt{username}
and \texttt{password} fields in the \texttt{server} blocks must of
course be edited with your user's specific correct values.

\lstinputlisting[basicstyle=\tiny,label=code:mvnsettings,caption=An example \texttt{\$HOME/.m2/settings.xml} file]{code/settings.xml}

\subsection{Git}
\subsubsection{Installation}

To install git and verify it's been correctly installed, please run the
commands in code listing \ref{code:gitinstall}.

\begin{lstlisting}[label=code:gitinstall,caption=Installing git and verifying it works]
# apt-get install git
$ git --version
git version 1.8.1.2
\end{lstlisting}

\subsubsection{Configuration}

Create an \texttt{\$HOME/.netrc} file with contents as in the example in
code listing \ref{code:gitauth}. You must of course enter your user's
specific correct values in the \texttt{login} and \texttt{password}
fields.

\begin{lstlisting}[label=code:gitauth,caption=An example \texttt{\$HOME/.netrc} file]
machine scm.acme.com
login username
password secret
\end{lstlisting}

\subsection{MySQL}
\subsubsection{Installation}

To install MySQL server and the command line client, please run the
commands in code listing \ref{code:mysqlinstall}.

\begin{lstlisting}[label=code:mysqlinstall,caption=Installing MySQL]
# apt-get install mysql-server
\end{lstlisting}

During the installation process, you will be asked for a root password:
this is the password that will be used for the MySQL admin user and has
nothing to do with the \ac{OS} root password.

To connect to MySQL server with the root user and to verify that the
installation was successful, run the commands in code listing
\ref{code:mysqlconnroot}. You should be asked for the root password,
then see the MySQL prompt.

\begin{lstlisting}[label=code:mysqlconnroot,caption=Connecting to MySQL server as the root user]
$ mysql -uroot -p mysql
\end{lstlisting}

\subsubsection{Configuration}

The default configuration should be okay. Just make sure the
\texttt{bind-address} directive\footnote{located in
\texttt{/etc/mysql/my.cnf}} specifies a correct ip address
value\footnote{the default of \texttt{127.0.0.1} is okay if you run the
MySQL server and the space rocket in the same machine, otherwise you
will need to specify an ip address that the space rocket can use to
remotely connect to the MySQL server}. 

\subsection{Node.js}
\subsubsection{Installation}

The Node.js tarball should be downloaded from the official download
page\footnote{\href{http://nodejs.org/download/}{\texttt{http://nodejs.org/download/}}}.
At the time of writing, the current version is 0.10.5.

To unpack the distribution tarball and verify its contents, please run
the commands in code listing \ref{code:nodeunpack}.

\begin{lstlisting}[label=code:nodeunpack,caption=Unpacking the tarball to the \texttt{local} directory and checking its contents]
$ tar xz -C ~/local -f ~/dist/node-v0.10.5-linux-x64.tar.gz 
$ ls ~/local/node-v0.10.5-linux-x64
bin  ChangeLog  lib  LICENSE  README.md  share
\end{lstlisting}

Modify the \texttt{\$HOME/.profile} file so that its bottom looks like
the example in code listing \ref{code:nodepath}.

\begin{lstlisting}[label=code:nodepath,caption=Code to add to the bottom of the \texttt{.profile} file]
JAVA_HOME="$HOME/local/jdk1.7.0_21"
SBT_HOME="$HOME/local/sbt"
SBT_OPTS='-XX:+CMSClassUnloadingEnabled -XX:MaxPermSize=256M'
MAVEN_HOME="$HOME/local/apache-maven-3.0.5"
NODE_HOME="$HOME/local/node-v0.10.5-linux-x64"
PATH="$NODE_HOME/bin:$MAVEN_HOME/bin:$SBT_HOME/bin:$JAVA_HOME/bin:$PATH"
\end{lstlisting}

Having modified the \texttt{.profile} file, run the commands in code
listing \ref{code:nodeverify} to make the changes available to the
current prompt and to verify that node\footnote{\texttt{node} is the
javascript runtime} and npm\footnote{\texttt{npm} is the node package
manager} work.

\begin{lstlisting}[label=code:nodeverify,caption=Sourcing the \texttt{.profile} file and verifying that node and npm work]
$ source ~/.profile
$ node --version
v0.10.5
$ npm --version
1.2.18
\end{lstlisting}

\subsection{Further notes}

The \texttt{.profile} file gets automatically sourced at every new login so there's no need to manually source it from now on.
