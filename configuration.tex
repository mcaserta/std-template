\section{Configuration}

\subsection{Logging}
\label{sec:logging}

All of the \ac{AP} software modules make use of a logging framework called logback\footnote{\href{http://logback.qos.ch/}{\texttt{http://logback.qos.ch/}}}. 

Logback configuration is performed through the \texttt{logback.xml} file. We will later provide the specific location of the \texttt{logback.xml} file for each software module in this section.

Full documentation regarding logback configuration can be found online\footnote{\href{http://logback.qos.ch/manual/configuration.html}{\texttt{http://logback.qos.ch/manual/configuration.html}}}. We will however go through an example for your convenience.

\lstinputlisting[basicstyle=\tiny,numbers=left,numbersep=5pt,numberstyle=\tiny\color{gray},xleftmargin=15pt,framexleftmargin=15pt,label=code:logbackxml,caption=An
example \texttt{logback.xml} configuration file]{code/logback.xml}

As you can see in code listing \ref{code:logbackxml}, there's a lot going on here. Let's see what's happening, line by line.

\begin{itemize}
\item[2] the configuration file should be automatically scanned for changes every 30 seconds and reloaded if necessary
\item[3] a file appender is declared
\item[4] the appender in 3 writes to a file called \texttt{space-rocket.log} in the \texttt{logs} directory
\item[5] a time based rolling policy appender is declared
\item[7] the appended in 5 will write files using the pattern specified here
\item[9] files older than 30 days are automatically deleted
\item[12] declares the pattern that will be used when printing to the appender in 3
\item[15] an appender that writes to the standard output is declared
\item[17] declares the pattern that will be used when printing to the appender in 15
\item[20] declares the root level
\item[21] attaches the appender in 3 to the root logger
\item[22] attaches the appender in 15 to the root logger
\item[24,25] the \texttt{com.acme} logger category is configured at \texttt{DEBUG} level
\item[27,28] the \texttt{org.mylin} logger category is configured at \texttt{INFO} level
\end{itemize}
